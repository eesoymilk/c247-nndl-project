\documentclass{article}


% if you need to pass options to natbib, use, e.g.:
%     \PassOptionsToPackage{numbers, compress}{natbib}
% before loading neurips_2024


\usepackage[final]{neurips_2024}

\usepackage[utf8]{inputenc} % allow utf-8 input
\usepackage[T1]{fontenc}    % use 8-bit T1 fonts
\usepackage{hyperref}       % hyperlinks
\usepackage{url}            % simple URL typesetting
\usepackage{booktabs}       % professional-quality tables
\usepackage{amsfonts}       % blackboard math symbols
\usepackage{nicefrac}       % compact symbols for 1/2, etc.
\usepackage{microtype}      % microtypography
\usepackage{xcolor}         % colors
\usepackage{graphicx}       % figures


\title{C247 Nerual Nework and Deep Learning\\Project Writeup}


% The \author macro works with any number of authors. There are two commands
% used to separate the names and addresses of multiple authors: \And and \AND.
%
% Using \And between authors leaves it to LaTeX to determine where to break the
% lines. Using \AND forces a line break at that point. So, if LaTeX puts 3 of 4
% authors names on the first line, and the last on the second line, try using
% \AND instead of \And before the third author name.


\author{%
  Yu-Wei Chang, Leyi Zou \\
  Department of Electrical and Computer Engineering\\
  University of California, Los Angeles\\
  Los Angeles, CA 90095 \\
  \texttt{\{ywchang,zelozou\}@ucla.edu} \\
}


\begin{document}


\maketitle


\begin{abstract}
  The abstract paragraph should be indented \nicefrac{1}{2}~inch (3~picas) on
  both the left- and right-hand margins. Use 10~point type, with a vertical
  spacing (leading) of 11~points.  The word \textbf{Abstract} must be centered,
  bold, and in point size 12. Two line spaces precede the abstract. The abstract
  must be limited to one paragraph.
\end{abstract}


\section{Introduction}
Our approach is primarily motivated by the discussions of recurrent neural networks (RNNs) and related architectures, 
including long short-term memory (LSTM) networks and gated recurrent units (GRUs), 
which are widely used for sequential data like sEMG signals. 
We further explore a hybrid model that integrates temporal convolutional networks (TCNs) with LSTM and GRU. 
Additionally, we investigate the impact of data quality on model performance by employing various preprocessing techniques 
and evaluating their effects on model accuracy and robustness.

\subsection{Architectures}

Surface electromyography (sEMG) signals provide a noninvasive means of detecting muscle activity in the forearms and wrists. However, sEMG data can be both noisy and highly variable across different recording sessions, electrodes, and users. This variability poses a challenge for sequence-level tasks such as keystroke decoding, in which the goal is to accurately predict typed characters from continuous muscle activity signals. To address these challenges, we explore four architectural paradigms, each designed to handle different aspects of the sEMG decoding problem.

The first one is the baseline structure by Hannun et al. (2019). It leverages Time-Depth Separable (TDS) convolutions to capture local temporal features while keeping the parameter footprint small. A short multi-layer perceptron (MLP) is also included to handle electrode shifts in a rotation-invariant manner. Although this approach has shown promise in prior work, it may not fully capture longer-range dependencies or complex spatial-temporal patterns in sEMG data.

The second structure is a convolutional networks tailored to time-series—often referred to as Temporal Convolutional Networks—enable flexible receptive fields through dilations, potentially modeling a wide range of time scales without explicit recurrent gating. By applying rotation-invariant convolutions across electrode channels, the TCN aims to address local, shift-related variability while exploring multiple scales in the time dimension.

Thirdly, purely recurrent designs (e.g., LSTM or GRU stacks) remain a key method for sequence modeling in speech, EEG, and now sEMG contexts, given their ability to learn long-term temporal dependencies. We employ a multi-band approach that processes each sEMG band independently, flattening the spectral-electrode features at each time step and passing them through LSTM and GRU layers in sequence. This design removes all convolutional blocks, thus providing an alternative baseline for purely RNN-based feature extraction.

Finally, we combine the strengths of dilated convolutions (fast, parallelizable extraction of local time patterns) with recurrent gating (robust modeling of long-range dependencies). A TCN front end handles initial band-wise, rotation-invariant feature extraction; the resulting timewise features are then processed by LSTM and GRU layers to refine the global temporal context.

All four architectures feed their per-timestep logits into a CTC loss, which aligns predictions to unlabeled time steps and enables end-to-end training. By comparing these approaches, we seek to identify which combination of local convolutional or global recurrent modeling best suits the inherently noisy and user-dependent nature of sEMG data for keystroke decoding.

\subsection{Preprocessing Motivations}

Based on empirical observations and prior research, 
we tested three preprocessing techniques to evaluate their effects on model performance.

First, considering that sEMG signals primarily fall within the frequency range of 0--500 Hz, 
we applied a bandpass filter to retain relevant signal components while eliminating out-of-band noise. 
Additionally, a notch filter was implemented to remove power line interference at 50 Hz. 
The combination of these two filters ensures that the acquired signals are free from common noise artifacts.

Second, we applied adaptive Gaussian noise for data augmentation. 
This technique enhances data diversity and improves the model's robustness to noise. 
Instead of introducing a fixed amount of noise, we adaptively adjust the noise level 
based on the standard deviation of the original data. 
This ensures that the introduced noise is proportional to the signal amplitude, 
so that the noise will not overwhelm some of the data, preserving the underlying signal characteristics.

Finally, we applied z-score normalization to standardize the data by removing the mean and scaling it to unit variance. 
This step enhances model stability by ensuring consistent feature scaling across different samples.




\section{Methods}

\subsection{Architectures}

In this section, we detail the four architectures employed for sEMG-based keystroke recognition, each of which outputs frame-level character probabilities for Connectionist Temporal Classification (CTC). All architectures take the same preprocessed inputs—two sEMG bands (left and right wrist), each with 16 electrode channels, transformed into spectrogram features—and produce per-timestep logits over the typing vocabulary plus a CTC blank token.

\subsubsection{Baseline: TDSConv}

We adopt the TDSConvCTCModule described in prior work as our baseline. Here, the model applies:

\begin{enumerate}
    \item\textbf{Spectrogram Normalization}: Normalize the amplitude distribution across channels.

    \item\textbf{Multi-Band Rotation Invariant MLP}: Processes each band (left, right) independently, applying a small multi-layer perceptron invariant to electrode-channel shifts.

    \item\textbf{TDS Convolutional Encoder}: It is composed of Time-Depth Separable (TDS) convolutional blocks [Hannun et al., 2019], each featuring a temporal 2D convolution (over time × features) and a pointwise fully connected residual layer.
    
    \item\textbf{Linear Layer}: A final linear layer maps the hidden representations to the desired output dimension, followed by a log-softmax activation.
\end{enumerate}

We include this TDSConv setup for comparison with our proposed alternatives, but concentrate primarily on the latter approaches.

\subsubsection{TCN-Based Architecture}

Our first custom model, TCNCTCModule, replaces TDSConv blocks with a Temporal Convolutional Network (TCN):

\begin{enumerate}
    \item\textbf{Spectrogram Normalization}: Normalize the inputs as before.

    \item\textbf{Multi-Band Rotational Invariant TCN}: Input spectrograms are split by band, with each band processed by a rotation-invariant TCN block. Each block rotates electrode channels by various offsets and takes a mean or max over these rotations to ensure invariance to minor electrode shift.

    \item\textbf{Dilated Convolutions}: The TCN employs increasing dilation factors to capture long-range dependencies in the time dimension.
    
    \item\textbf{Flatten + Linear}: After TCN feature extraction, we flatten the band features per timestep and apply a final linear projection to the character classes, followed by log-softmax.
\end{enumerate}

Unlike the baseline’s TDS convolution, the TCN design uses 1D convolutions over time with flexible receptive fields, potentially offering more direct parallelization and multi-scale temporal modeling.

\subsubsection{LSTM+GRU Model}

Our second custom model, LSTMGRUCTCModule, removes the convolutional encoder entirely and relies on recurrent networks to handle all temporal structure:
\begin{enumerate}
    \item\textbf{Spectrogram Normalization}: Normalize the inputs as before.

    \item \textbf{Multi-Band LSTM + GRU}: Each sEMG band is flattened into (electrodes × freq) features and fed into a small LSTM, optionally followed by dropout, and then a GRU. We do this per band, outputting band-specific hidden states.

    \item \textbf{Flatten + Linear}: The time dimension remains intact, so we simply flatten across bands and feed these hidden states into a final linear layer. A log-softmax produces per-timestep character logits.
\end{enumerate}

By combining LSTM and GRU, we aim to capture nuanced temporal dependencies without any convolution, thus contrasting with the TCN and TDSConv approaches.

\subsubsection{Hybrid TCN + LSTM + GRU}

Lastly, we propose HybridCTCModule, which fuses the TCN and LSTM+GRU concepts:

\begin{enumerate}
    \item\textbf{Spectrogram Normalization}: Normalize the inputs as before.

    \item\textbf{Multi-Band Rotation Invariant TCN}: Each frequency band is handled separately, capturing mid-range temporal dynamics via dilated convolutions.

    \item\textbf{Flatten}: Merge the per-band TCN outputs into a sequence representation.

    \item\textbf{LSTM + GRU}: Apply previous LSTM + GRU, on top of the TCN outputs to further refine temporal context at longer timescales.

    \item\textbf{Linear}: It maps the resultant hidden states to output classes, with a log-softmax.
\end{enumerate}

This hybrid approach integrates convolutional parallelism with recurrent long-term memory, potentially extracting richer spatiotemporal patterns than either method in isolation.


\subsection{Data Preprocessing}

We primarily explored three forms of preprocessing: combination filtering (bandpass and notch), 
adaptive Gaussian noise, and z-score normalization.
They were tested separately on the baseline model to evaluate their effects on model performance.

\subsubsection{Combination Filtering}

For tests involving filters, the temporal data was successively passed through a Butterworth bandpass filter and a notch filter (Table~\ref{filters}).
The sampling frequency for both filters was set to 2000 Hz, which is the same as the sampling frequency of the original data,
so the filter design was based on the Nyquist frequency of 1000 Hz.
Both filters were designed using forward-backward filtering to avoid phase distortion.
We set the order of the bandpass filter to 4 and the cutoff frequencies to 20 Hz and 150 Hz / 500 Hz, respectively.
Then, we applied another 2-order notch filter.
The Q factor of the notch filter was set to 30 and the middle frequency was set to 50 Hz, 
which are common values in this context. This configuration ensures a narrow bandwidth to effectively remove the power line interference.
Thus, we applied a relatively steep filter to remove signal components outside the range of 20-500 Hz 
while also eliminating power line interference at 50 Hz.

\begin{table}
    \caption{Filters}
    \label{filters}
    \centering
    \begin{tabular}{lll}
      \toprule
      \cmidrule(r){1-2}
      Type of Filter     & Frequency Range (Hz)     & Reference \\
      \midrule
      $4^{th}$ Butterworth bandpass filter & 20 - 150/500 Hz  &  ~[1], ~[3]    \\
      $2^{nd}$ Butterworth notch filter    & 50 Hz        &  ~[1], ~[2]    \\
      \bottomrule
    \end{tabular}
  \end{table}
  
\subsubsection{Adaptive Gaussian Noise}

For the adaptive Gaussian noise, we added noise to the data based on the standard deviation of the original data.
Specifically, we set the noise ratio to 0.05, which is a relatively small value, to ensure that the noise does not overwhelm the original signal.
For a given sample, we generated Gaussian noise with a mean of 0 and a standard deviation of 0.05 times the standard deviation of the original data
along the time axis to add to the diversity of the data.

\subsubsection{Z-score Normalization}

For normalization, we applied z-score normalization along the time axis to 
standardize the data by subtracting the mean and scaling it to unit variance.

\subsubsection{Preprocessing Pipeline}

These steps were arranged in the order of filtering, noise addition, and normalization.
This is because general noise and artifacts do not carry useful information, so they are removed first.
The noise addition is adaptive to the intensity of original data, so it is better to add it before normalization.
Finally, the normalization step is applied to ensure that the data is standardized and ready for model training.
After these steps, we implemented the random band rotation, temporal alignment jitter, 
log spectrogram and specaugment as in the baseline model.
It is worth noting that the filtering and normalization should also be applied to the validation and test sets to ensure consistency in the data preprocessing pipeline.


\section{Results}

\subsection{Architectures}

\begin{figure}
    \centering
    \begin{subfigure}[b]{0.3\textwidth}
        \centering
        \includegraphics[width=\textwidth]{figures/all_losses.png}
        \caption{Training losses for all architectures.}
        \label{fig:y equals x}
    \end{subfigure}
    \hfill
    \begin{subfigure}[b]{0.3\textwidth}
        \centering
        \includegraphics[width=\textwidth]{figures/Val_cer.png}
        \caption{Validation CER for all architectures.}
    \end{subfigure}
    \hfill
    \begin{subfigure}[b]{0.3\textwidth}
        \centering
        \includegraphics[width=\textwidth]{figures/test.png}
        \caption{Test CER for all architectures.}
    \end{subfigure}
    \caption{Model performance comparison.}
    \label{fig:result}
\end{figure}

All four neural architectures were trained for a total of $200$ epochs. Due to computational constraints, we partitioned the training into two consecutive phases of $100$ epochs each. We checkpointed each model after the initial $100$ epochs, resuming the second phase of training from these checkpoints. To differentiate clearly between these two training stages, we appended the suffix \texttt{\_200} to denote metrics collected during the second $100$ epochs.

\subsubsection{Baseline (TDSConv) Model.}

The baseline model, adapted directly from Hannun et al., consistently achieved the strongest performance among all architectures tested. Throughout the $200$ epochs, its training loss exhibited a steady, continuous decline, suggesting that the network had not yet reached full convergence. Furthermore, the validation Character Error Rate (CER) similarly reflected persistent improvement, reinforcing the baseline’s effectiveness at generalizing beyond the training set. This ongoing downward trend in both metrics indicates that, given extended training epochs or additional computational resources, the baseline may yield further performance gains. Such strong performance underscores the suitability of Time-Depth Separable convolutions for capturing the underlying patterns present in sEMG signals for keystroke prediction.

\subsubsection{Temporal Convolutional Network (TCN).}

The TCN-based architecture showed promising early training dynamics, characterized by rapid decreases in loss, mirroring the baseline model’s trajectory. We attribute this similarity to the shared convolutional backbone between TCN and TDSConv, allowing both models to efficiently extract local temporal features from sEMG inputs. However, despite the initially impressive loss reduction, TCN began showing signs of overfitting in later training stages. This phenomenon was evident from the plateauing and subsequent slight deterioration in the validation CER, despite continued reductions in the training loss. This discrepancy indicates that the TCN model effectively memorized subtle nuances in the training dataset, but struggled to generalize well to unseen data. The early promise and later challenges of the TCN architecture suggest the need for further regularization techniques, such as dropout or weight decay, to better manage model capacity and enhance generalization.

\subsubsection{LSTM+GRU Model.}

Our recurrent-only approach, composed of stacked LSTM and GRU layers, presented a distinctly different training profile compared to convolutional architectures. Initially, the recurrent model demonstrated slower convergence in terms of loss reduction. Such gradual initial convergence aligns with expectations, as recurrent neural networks typically require extended training periods to effectively learn long-range temporal dependencies. Encouragingly, towards the end of the $200$ epochs, the LSTM+GRU model still exhibited a persistent, albeit slow, decline in both training loss and validation CER, indicating continued learning potential. The ongoing improvements observed at the conclusion of training strongly suggest that the model had not reached its optimal performance and could further benefit from prolonged training or more aggressive optimization strategies.

\subsubsection{Hybrid (TCN+LSTM+GRU) Model.}

The hybrid architecture, intended to leverage the strengths of both convolutional and recurrent structures, exhibited a training profile similar to, though slightly inferior than, the purely recurrent LSTM+GRU approach. Despite early expectations that combining local convolutional feature extraction (via TCN) and global temporal context modeling (via LSTM and GRU layers) might yield superior performance, the observed results did not fully realize these advantages. We hypothesize that introducing convolutional components into a recurrent pipeline increased overall model complexity, thereby posing additional challenges in training optimization. Specifically, the hybrid model might not have fully exploited the richer representations generated by the TCN front end within the available computational resources and training time. Consequently, additional hyperparameter tuning or the introduction of intermediate regularization measures may be necessary to unlock the hybrid model’s full potential.

\subsubsection{Summary and Discussion}

In summary, the baseline TDSConv model consistently outperformed our alternative architectures, showcasing a robust and reliable capability to decode keystrokes from sEMG signals. The purely convolutional TCN architecture shared rapid initial convergence characteristics with TDSConv, yet suffered from overfitting later in training. Conversely, the purely recurrent LSTM+GRU model, although slower to converge initially, continued improving through the entirety of training, indicating substantial room for growth with further optimization. Finally, the hybrid TCN+LSTM+GRU model, while theoretically appealing, faced optimization difficulties due to increased complexity and thus did not outperform simpler approaches. Future research directions may explore extended training duration, advanced regularization methods, targeted data augmentation strategies, or further hyperparameter tuning, to better capitalize on the strengths of each proposed model.

\subsection{Data Preprocessing}

We separately tested the effects of three preprocessing techniques in 2.2 on the baseline model's performance.
In view of limited computational resources, we cannot fully optimize all the techniques,
so we only test a few hyperparameters for the techniques, especially for the filtering.

\subsubsection{Experiment Settings}
First, we applied a notch filter to remove power line interference at 50 Hz.
Upon investigating previous works in sEMG signal processing, we found that the frequency range of sEMG signals is primarily between 20 Hz and 150 Hz.
Therefore, we trained the model (with the notch filter already added) with a bandpass filter with a cutoff frequency of 20 Hz and 500 Hz or with another bandpass filter with a cutoff frequency of 20 Hz and 150 Hz.
We trained the three models and compared their performance at the end of Epoch 50.

For the adaptive Gaussian noise, we fixed the noise ratio  to 0.05, which is a relatively small value.
Because of time constraints, we did not test other values.

\subsubsection{Results}

\begin{figure}
    \centering
    \includegraphics[width=0.8\textwidth]{figures/data_train.png}
    \caption{Comparison of the CER of the baseline model and the models with preprocessing techniques}
    \label{fig:data_train}
\end{figure}
  

As shown in \ref{fig:data_train}, we found that the validation CER of model with the notch filter surpassed the baseline model by 8.2\% at the end of Epoch 50.
The model with adaptive Gaussian noise and the model with z-score normalization also achieved a CER of 22.13 and 22.37 respectively, 
which is quite close to the baseline model's CER of 20.98.
However, the model with the bandpass filters both performed poorly, while they had a very high CER at the beginning of training.

It is also worth noting that the models with any of the preprocessing techniques performed much better than the baseline model at the beginning of training, 
all of which had a CER that was 5 times lower than the baseline model.
This indicates that the preprocessing techniques can help the model converge faster.
The models with bandpass filters and z-score normalization especially performed well at the beginning of training.



\section{Discussion}

\subsubsection{Model Architecture}
An important context for the results is that the data used was from a single user, so the data volume and diversity are limited. The validation and test loss of the hybrid and LSTM-GRU models are significantly lower than those of the other models, while the TCN model has the highest loss. This may suggest that the TCN model is poorly suited for this task. For sEMG, which may require capturing signals with long-term dependencies or strong individual differences, more convolutional layers or more sophisticated parameter adjustments may be required. The TCN model converges too quickly, which may indicate that it is overfitting the training data. On the other hand, the hybrid model and LSTM-GRU model are better at capturing the temporal dependencies in the data. However, because our dataset is limited to a single user, models like the hybrid model may not fully utilize their potential. Multi-user datasets could better exploit the advantages of these models, but they would also increase the challenge of generalization.

Another observation is that the hybrid model took longer to converge than the baseline model. This may be due to the complexity of the hybrid model, and it also suggests that the hybrid model is more prone to overfitting.

\subsubsection{Preprocessing Techniques}
The results of our experiments showed that the model using the 50 Hz notch filter achieved a validation CER that was 8.2\% lower than the baseline model at Epoch 50. This may indicate that removing power line interference is crucial for processing sEMG signals. After removing the 50 Hz interference, the model can focus more on the effective frequency components in the signal, thereby extracting more meaningful features and improving the final performance.

Adaptive Gaussian noise and z-score normalization did not significantly surpass the baseline in the final CER (22.13 and 22.37, respectively, close to the baseline's 20.98), but they performed significantly better than the baseline model in the early stages of training. This may be because normalization and noise addition reduce the model's sensitivity to the scale and distribution of the input data in the early stages, allowing the network to learn the main patterns in the signal faster, thus accelerating convergence. However, in the later stages, the advantages of these preprocessing methods may be balanced by other factors, resulting in final performance that is close to the baseline.

When two bandpass filters with different cutoff frequencies (20 Hz–500 Hz and 20 Hz–150 Hz) were used, the model had very high CER at the beginning of training and poor overall performance. This may indicate that the bandpass filter failed to retain enough of the effective signal components, or there were side effects in the filter design that resulted in the loss of useful information. The much lower CER of these models can be attributed to the bandpass filters suppressing noise, but they may have also weakened some important detailed features that help the model learn. The hyperparameters of the bandpass filter for this task need further optimization to achieve better performance.

From the results, it is evident that the preprocessing techniques can help the model converge faster to some extent. This also suggests that appropriate preprocessing techniques can play an important role in improving learning efficiency.


\section*{References}


% References follow the acknowledgments in the camera-ready paper. Use unnumbered first-level heading for
% the references. Any choice of citation style is acceptable as long as you are
% consistent. It is permissible to reduce the font size to \verb+small+ (9 point)
% when listing the references.
% Note that the Reference section does not count towards the page limit.
% \medskip


{
\small
\begin{enumerate}[label={[}\arabic*{]}]
    \item Shehla, I.,\  Haramain, S.A.,\ Shafique, S.,\  Rabail, A.,\  Amin, F.,\  Waqar, M.,\  \& Afzal, M.\ (2021) 
    A Brief Review of Strategies Used for EMG Signal Classification.  
    In \textit{Proceedings of the 2021 International Conference on Artificial Intelligence (ICAI)},  
    Islamabad, Pakistan, April 05-07, 2021. IEEE.  
    DOI: 10.1109/ICAI52063.2021.9435257.  


    \item Merletti, R.\ \& Cerone, G.L.\
    (2020) Tutorial. Surface EMG detection, conditioning and pre-processing: Best 
    practices.
    {\it Journal of Electromyography and Kinesiology} {\bf 54}(2020):102440.


    \item Phinyomark, A., Phukpattaranont, P.\ \& Limsakul, C.\
    (2012) Feature reduction and selection for EMG signal classification.
    {\it Expert Systems with Applications} {\bf 39}(2012):7420-7431.
    
    \item Lea, C., Vidal, R., Reiter, A., \& Hager, G. D. (2016). Temporal convolutional networks: A unified approach to action segmentation. In Computer vision–ECCV 2016 workshops: Amsterdam, the Netherlands, October 8-10 and 15-16, 2016, proceedings, part III 14 (pp. 47-54). Springer International Publishing.

    \item S. Hochreiter and J. Schmidhuber, "Long Short-Term Memory," in Neural Computation, vol. 9, no. 8, pp. 1735-1780, 15 Nov. 1997, doi: 10.1162/neco.1997.9.8.1735.

    \item Chung, J., Gulcehre, C., Cho, K., \& Bengio, Y. (2014). Empirical evaluation of gated recurrent neural networks on sequence modeling. arXiv preprint arXiv:1412.3555.
\end{enumerate}
}





\appendix

\section{Appendix / supplemental material}

\begin{table}
    \caption{Hyperparameters for TCNCTCModule}
    \label{tab:tcn_ctc}
    \centering
    \begin{tabular}{ll}
      \toprule
      \textbf{Hyperparameter} & \textbf{Value} \\
      \midrule
      num\_filters & [128, 128, 128, 256] \\
      kernel\_size & 32 \\
      dilation\_base & 3 \\
      \bottomrule
    \end{tabular}
    \label{tab:tcn_hyperparameters}
\end{table}


\begin{table}
    \caption{Hyperparameters for LSTMGRUCTCModule}
    \label{tab:lstm_gru_ctc}
    \centering
    \begin{tabular}{ll}
      \toprule
      \textbf{Hyperparameter} & \textbf{Value} \\
      \midrule
      lstm\_layers & 2 \\
      lstm\_hidden\_size & 192 \\
      lstm\_dropout & 0.3 \\
      between\_dropout & 0.3 \\
      gru\_layers & 1 \\
      gru\_hidden\_size & 64 \\
      gru\_bidirectional & true \\
      \bottomrule
    \end{tabular}
    \label{tab:lstm_gru_hyperparameters}
\end{table}

\begin{table}
    \caption{Hyperparameters for HybridCTCModule}
    \label{tab:hybrid_ctc}
    \centering
    \begin{tabular}{ll}
      \toprule
      \textbf{Hyperparameter} & \textbf{Value} \\
      \midrule
      num\_filters & [128, 128, 128, 256] \\
      kernel\_size & 32 \\
      dilation\_base & 3 \\
      lstm\_layers & 2 \\
      lstm\_hidden\_size & 192 \\
      lstm\_dropout & 0.3 \\
      between\_dropout & 0.3 \\
      gru\_layers & 1 \\
      gru\_hidden\_size & 64 \\
      gru\_bidirectional & true \\
      \bottomrule
    \end{tabular}
    \label{tab:hybrid_hyperparameters}
\end{table}

% \input{template_guide.tex}
%
% \input{paper_checklist.tex}

\end{document}
