
\section*{References}


% References follow the acknowledgments in the camera-ready paper. Use unnumbered first-level heading for
% the references. Any choice of citation style is acceptable as long as you are
% consistent. It is permissible to reduce the font size to \verb+small+ (9 point)
% when listing the references.
% Note that the Reference section does not count towards the page limit.
% \medskip


{
\small
\begin{enumerate}[label={[}\arabic*{]}]
    \item Shehla, I.,\  Haramain, S.A.,\ Shafique, S.,\  Rabail, A.,\  Amin, F.,\  Waqar, M.,\  \& Afzal, M.\ (2021) 
    A Brief Review of Strategies Used for EMG Signal Classification.  
    In \textit{Proceedings of the 2021 International Conference on Artificial Intelligence (ICAI)},  
    Islamabad, Pakistan, April 05-07, 2021. IEEE.  
    DOI: 10.1109/ICAI52063.2021.9435257.  


    \item Merletti, R.\ \& Cerone, G.L.\
    (2020) Tutorial. Surface EMG detection, conditioning and pre-processing: Best 
    practices.
    {\it Journal of Electromyography and Kinesiology} {\bf 54}(2020):102440.


    \item Phinyomark, A., Phukpattaranont, P.\ \& Limsakul, C.\
    (2012) Feature reduction and selection for EMG signal classification.
    {\it Expert Systems with Applications} {\bf 39}(2012):7420-7431.
    
    \item Lea, C., Vidal, R., Reiter, A., \& Hager, G. D. (2016). Temporal convolutional networks: A unified approach to action segmentation. In Computer vision–ECCV 2016 workshops: Amsterdam, the Netherlands, October 8-10 and 15-16, 2016, proceedings, part III 14 (pp. 47-54). Springer International Publishing.

    \item S. Hochreiter and J. Schmidhuber, "Long Short-Term Memory," in Neural Computation, vol. 9, no. 8, pp. 1735-1780, 15 Nov. 1997, doi: 10.1162/neco.1997.9.8.1735.

    \item Chung, J., Gulcehre, C., Cho, K., \& Bengio, Y. (2014). Empirical evaluation of gated recurrent neural networks on sequence modeling. arXiv preprint arXiv:1412.3555.
\end{enumerate}
}


