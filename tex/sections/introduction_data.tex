\subsection{Preprocessing Motivations}

Based on empirical observations and prior research, 
we tested three preprocessing techniques to evaluate their effects on model performance.

First, considering that sEMG signals primarily fall within the frequency range of 0--500 Hz, 
we applied a bandpass filter to retain relevant signal components while eliminating out-of-band noise. 
Additionally, a notch filter was implemented to remove power line interference at 50 Hz. 
The combination of these two filters ensures that the acquired signals are free from common noise artifacts.

Second, we applied adaptive Gaussian noise for data augmentation. 
This technique enhances data diversity and improves the model's robustness to noise. 
Instead of introducing a fixed amount of noise, we adaptively adjust the noise level 
based on the standard deviation of the original data. 
This ensures that the introduced noise is proportional to the signal amplitude, 
so that the noise will not overwhelm some of the data, preserving the underlying signal characteristics.

Finally, we applied z-score normalization to standardize the data by removing the mean and scaling it to unit variance. 
This step enhances model stability by ensuring consistent feature scaling across different samples.

