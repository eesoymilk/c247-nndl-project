\subsection{Data Preprocessing}

We primarily explored three forms of preprocessing: combination filtering (bandpass and notch), 
adaptive Gaussian noise, and z-score normalization.
They were tested separately on the baseline model to evaluate their effects on model performance.

\subsubsection{Combination Filtering}

For tests involving filters, the temporal data was successively passed through a Butterworth bandpass filter and a notch filter (Table~\ref{filters}).
The sampling frequency for both filters was set to 2000 Hz, which is the same as the sampling frequency of the original data,
so the filter design was based on the Nyquist frequency of 1000 Hz.
Both filters were designed using forward-backward filtering to avoid phase distortion.
We set the order of the bandpass filter to 4 and the cutoff frequencies to 20 Hz and 150 Hz / 500 Hz, respectively.
Then, we applied another 2-order notch filter.
The Q factor of the notch filter was set to 30 and the middle frequency was set to 50 Hz, 
which are common values in this context. This configuration ensures a narrow bandwidth to effectively remove the power line interference.
Thus, we applied a relatively steep filter to remove signal components outside the range of 20-500 Hz 
while also eliminating power line interference at 50 Hz.

\begin{table}
    \caption{Filters}
    \label{filters}
    \centering
    \begin{tabular}{lll}
      \toprule
      \cmidrule(r){1-2}
      Type of Filter     & Frequency Range (Hz)     & Reference \\
      \midrule
      $4^{th}$ Butterworth bandpass filter & 20 - 150/500 Hz  &  ~[1], ~[3]    \\
      $2^{nd}$ Butterworth notch filter    & 50 Hz        &  ~[1], ~[2]    \\
      \bottomrule
    \end{tabular}
  \end{table}
  
\subsubsection{Adaptive Gaussian Noise}

For the adaptive Gaussian noise, we added noise to the data based on the standard deviation of the original data.
Specifically, we set the noise ratio to 0.05, which is a relatively small value, to ensure that the noise does not overwhelm the original signal.
For a given sample, we generated Gaussian noise with a mean of 0 and a standard deviation of 0.05 times the standard deviation of the original data
along the time axis to add to the diversity of the data.

\subsubsection{Z-score Normalization}

For normalization, we applied z-score normalization along the time axis to 
standardize the data by subtracting the mean and scaling it to unit variance.

\subsubsection{Preprocessing Pipeline}

These steps were arranged in the order of filtering, noise addition, and normalization.
This is because general noise and artifacts do not carry useful information, so they are removed first.
The noise addition is adaptive to the intensity of original data, so it is better to add it before normalization.
Finally, the normalization step is applied to ensure that the data is standardized and ready for model training.
After these steps, we implemented the random band rotation, temporal alignment jitter, 
log spectrogram and specaugment as in the baseline model.
It is worth noting that the filtering and normalization should also be applied to the validation and test sets to ensure consistency in the data preprocessing pipeline.